\documentclass[11pt]{article}
\usepackage{palatino}
\usepackage{fullpage}
\usepackage[font={it}]{caption}

\usepackage{amsmath}
\usepackage{amssymb}
\usepackage{mathpazo}
\usepackage{palatino}
\usepackage{graphicx}
\pagestyle{empty}
\usepackage{subfig}
\usepackage{comment}
\usepackage{algorithm}
\usepackage{booktabs}
\usepackage{algpseudocode}


\newcommand{\boldA}{\boldsymbol{A}}
\newcommand{\boldB}{\boldsymbol{B}}
\newcommand{\boldC}{\boldsymbol{C}}
\newcommand{\boldD}{\boldsymbol{D}}
\newcommand{\boldE}{\boldsymbol{E}}
\newcommand{\boldF}{\boldsymbol{F}}
\newcommand{\boldG}{\boldsymbol{G}}
\newcommand{\boldH}{\boldsymbol{H}}
\newcommand{\boldI}{\boldsymbol{I}}
\newcommand{\boldJ}{\boldsymbol{J}}
\newcommand{\boldK}{\boldsymbol{K}}
\newcommand{\boldL}{\boldsymbol{L}}
\newcommand{\boldM}{\boldsymbol{M}}
\newcommand{\boldN}{\boldsymbol{N}}
\newcommand{\boldO}{\boldsymbol{O}}
\newcommand{\boldP}{\boldsymbol{P}}
\newcommand{\boldQ}{\boldsymbol{Q}}
\newcommand{\boldR}{\boldsymbol{R}}
\newcommand{\boldS}{\boldsymbol{S}}
\newcommand{\boldT}{\boldsymbol{T}}
\newcommand{\boldU}{\boldsymbol{U}}
\newcommand{\boldV}{\boldsymbol{V}}
\newcommand{\boldW}{\boldsymbol{W}}
\newcommand{\boldX}{\boldsymbol{X}}
\newcommand{\boldY}{\boldsymbol{Y}}
\newcommand{\boldZ}{\boldsymbol{Z}}
\newcommand{\bolda}{\boldsymbol{a}}
\newcommand{\boldb}{\boldsymbol{b}}
\newcommand{\boldc}{\boldsymbol{c}}
\newcommand{\boldd}{\boldsymbol{d}}
\newcommand{\bolde}{\boldsymbol{e}}
\newcommand{\boldf}{\boldsymbol{f}}
\newcommand{\boldg}{\boldsymbol{g}}
\newcommand{\boldh}{\boldsymbol{h}}
\newcommand{\boldi}{\boldsymbol{i}}
\newcommand{\boldj}{\boldsymbol{j}}
\newcommand{\boldk}{\boldsymbol{k}}
\newcommand{\boldl}{\boldsymbol{l}}
\newcommand{\boldm}{\boldsymbol{m}}
\newcommand{\boldn}{\boldsymbol{n}}
\newcommand{\boldo}{\boldsymbol{o}}
\newcommand{\boldp}{\boldsymbol{p}}
\newcommand{\boldq}{\boldsymbol{q}}
\newcommand{\boldr}{\boldsymbol{r}}
\newcommand{\bolds}{\boldsymbol{s}}
\newcommand{\boldt}{\boldsymbol{t}}
\newcommand{\boldu}{\boldsymbol{u}}
\newcommand{\boldv}{\boldsymbol{v}}
\newcommand{\boldw}{\boldsymbol{w}}
\newcommand{\boldx}{\boldsymbol{x}}
\newcommand{\boldy}{\boldsymbol{y}}
\newcommand{\boldz}{\boldsymbol{z}}

\newcommand{\mcA}{\mathcal{A}}
\newcommand{\mcB}{\mathcal{B}}
\newcommand{\mcC}{\mathcal{C}}
\newcommand{\mcD}{\mathcal{D}}
\newcommand{\mcE}{\mathcal{E}}
\newcommand{\mcF}{\mathcal{F}}
\newcommand{\mcG}{\mathcal{G}}
\newcommand{\mcH}{\mathcal{H}}
\newcommand{\mcI}{\mathcal{I}}
\newcommand{\mcJ}{\mathcal{J}}
\newcommand{\mcK}{\mathcal{K}}
\newcommand{\mcL}{\mathcal{L}}
\newcommand{\mcM}{\mathcal{M}}
\newcommand{\mcN}{\mathcal{N}}
\newcommand{\mcO}{\mathcal{O}}
\newcommand{\mcP}{\mathcal{P}}
\newcommand{\mcQ}{\mathcal{Q}}
\newcommand{\mcR}{\mathcal{R}}
\newcommand{\mcS}{\mathcal{S}}
\newcommand{\mcT}{\mathcal{T}}
\newcommand{\mcU}{\mathcal{U}}
\newcommand{\mcV}{\mathcal{V}}
\newcommand{\mcW}{\mathcal{W}}
\newcommand{\mcX}{\mathcal{X}}
\newcommand{\mcY}{\mathcal{Y}}
\newcommand{\mcZ}{\mathcal{Z}}

\newcommand{\reals}{\ensuremath{\mathbb{R}}}
\newcommand{\integers}{\ensuremath{\mathbb{Z}}}
\newcommand{\rationals}{\ensuremath{\mathbb{Q}}}
\newcommand{\naturals}{\ensuremath{\mathbb{N}}}
\newcommand{\trans}{\ensuremath{\mathsf{T}}}
\newcommand{\ident}{\boldsymbol{I}}
\newcommand{\bzero}{\boldsymbol{0}}

\newcommand{\balpha}{\boldsymbol{\alpha}}
\newcommand{\bbeta}{\boldsymbol{\beta}}
\newcommand{\bdelta}{\boldsymbol{\delta}}
\newcommand{\boldeta}{\boldsymbol{\eta}}
\newcommand{\bkappa}{\boldsymbol{\kappa}}
\newcommand{\bgamma}{\boldsymbol{\gamma}}
\newcommand{\bmu}{\boldsymbol{\mu}}
\newcommand{\bphi}{\boldsymbol{\phi}}
\newcommand{\bpi}{\boldsymbol{\pi}}
\newcommand{\bpsi}{\boldsymbol{\psi}}
\newcommand{\bsigma}{\boldsymbol{\sigma}}
\newcommand{\btheta}{\boldsymbol{\theta}}
\newcommand{\bxi}{\boldsymbol{\xi}}
\newcommand{\bGamma}{\boldsymbol{\Gamma}}
\newcommand{\bLambda}{\boldsymbol{\Lambda}}
\newcommand{\bOmega}{\boldsymbol{\Omega}}
\newcommand{\bPhi}{\boldsymbol{\Phi}}
\newcommand{\bPi}{\boldsymbol{\Pi}}
\newcommand{\bPsi}{\boldsymbol{\Psi}}
\newcommand{\bSigma}{\boldsymbol{\Sigma}}
\newcommand{\bTheta}{\boldsymbol{\Theta}}
\newcommand{\bUpsilon}{\boldsymbol{\Upsilon}}
\newcommand{\bXi}{\boldsymbol{\Xi}}
\newcommand{\bepsilon}{\boldsymbol{\epsilon}}

\def\argmin{\operatornamewithlimits{arg\,min}}
\def\argmax{\operatornamewithlimits{arg\,max}}

\newcommand{\given}{\,|\,}
\newcommand{\distNorm}{\mathcal{N}}

\title{Optimization of Caloric and Macronutrient Intake through Constraint Satisfaction Problems}
\author{Jason Cui and Annie Hwang}
\begin{document}
\maketitle{}


\section{Introduction}

A description of the purpose, goals, and scope of your system or
empirical investigation.  You should include references to papers you
read on which your project and any algorithms you used are
based. Include a discussion of whether you adapted a published
algorithm or devised a new one, the range of problems and issues you
addressed, and the relation of these problems and issues to the
techniques and ideas covered in the course.
-----

Testing

The idea for the project came to be when we realized it was very difficult to keep track and be on top of your nutrition. Leading a healthy diet and lifestyle can be incredibly difficult, and with busy schedules and not enough time to cook it can be easy to overlook nutrition.


There exist software solutions out there designed to help make tracking nutrition easier. The best example is MyFitnessPal, which is a mobile phone application that allows users to input their daily intake of foods in order to keep track of calories and macronutrients. There also exist services like Yelp that allow users to discover restaurants and places to eat. The problem with these sorts of services, however, is that they are very low level. In the end, individuals still have to make their own decisions about where they eat and what they choose to eat.

existing algorithm usage: George Dantzig proposed a greedy approximation algorithm to solve the unbounded knapsack problem.

\section{Background and Related Work}

For instance, \cite{hochreiter1997long}.\par
\vspace{2mm}
\noindent There has been many papers and research done regarding the knapsack problem.... blah blah blah

\section{Problem Specification}
For this project, we wanted to come up with an intelligent solution to allow users to optimize their selection of food while satisfying named constraints. More specifically, we wanted to see if we could come up with a system that could not only keep track of caloric and macronutrient intake, but actually recommend individuals foods to eat based off of macronutrient intake limits (i.e. total fat, sugar, etc) and nutritional needs. \par
\vspace{2mm}
\noindent We mapped this to a CSP with an optimization component to it. Throughout the project we looked into 4 different approaches to optimize the objective function while satisfying the constraints in the CSP: 1) Greedy solution, 2) Local search solution, 3) Brute-force solution, and lastly, a 4) Dynamic programming solution. In the most simplest case with one objective function to maximize under a single constraint, we realized that this was essentially the knapsack problem. In this next section, we outline these four approaches and how they different in performance.


\section{Approach}

\textbf{Approach 1: Greedy Solution}

\noindent George Dantzig proposed a greedy approximation algorithm to solve the unbounded knapsack problem.[19] His version sorts the items in decreasing order of value per unit of weight, $\frac{ v_{i}}{w_{i}}$ It then proceeds to insert them into the sack, starting with as many copies as possible of the first kind of item until there is no longer space in the sack for more. Provided that there is an unlimited supply of each kind of item, if $m$ is the maximum value of items that fit into the sack, then the greedy algorithm is guaranteed to achieve at least a value of $m/2$. However, for the bounded problem, where the supply of each kind of item is limited, the algorithm may be far from optimal. 

\begin{algorithm}
  \begin{algorithmic}
    \Procedure{MyAlgorithm}{$b$}
    \State{knapsack = []}
    \State{foods = items sorted by $v_i/w_i$ in descending order}
    \While{$totalWeight < limit$}
    \State{knapsack.append(current food)}
    \State{totalWeight += current weight}
    \ENDWHILE  
    \Return {knapsack}
    \EndProcedure{}
  \end{algorithmic}
  \caption{Greedy Solution.}
\end{algorithm}


\begin{algorithm}
  \begin{algorithmic}
    \Procedure{MyAlgorithm}{$b$}
    \State{$a \gets 10$}
    \EndProcedure{}
  \end{algorithmic}
  \caption{Local Search 1- Hill Climbing.}
\end{algorithm}


\begin{algorithm}
  \begin{algorithmic}
    \Procedure{MyAlgorithm}{$b$}
    \State{$a \gets 10$}
    \EndProcedure{}
  \end{algorithmic}
  \caption{Local Search 2 - Simulated Annealing.}
\end{algorithm}

\noindent \textbf{Approach 4: Brute Force}

\noindent For the brute force method to solve this knapsack problem, we generated all combinations of food items and only considered the ones that satisfied the specified macronutrient weight limit. This has a run-time of $O(2^n)$, where $n$ is the number of items. It is extremely slow to run as the item number increases exponentially.
\vspace{2mm}
\begin{algorithm}
  \begin{algorithmic}
    \Procedure{MyAlgorithm}{$b$}
    \State{bestTotalValue = 0}
    \State{bestItems = None}
    \State{Generate all combinations of food item}
    
    \For{items in combination}
    \If {totalWeight of items $<$ limit \textbf{and} totalValue $>$ bestTotalValue}
    \State{$bestTotalValue \gets totalValue$}
    \State{$bestItems \gets items$}
    \ENDIF
    \ENDFOR
    
    \Return bestItems
    \EndProcedure{}
  \end{algorithmic}
  \caption{Brute Force}
\end{algorithm}

\noindent \textbf{Approach 5: Dynamic Programming}



\begin{algorithm}
  \begin{algorithmic}
    \Procedure{MyAlgorithm}{$b$}
    \State{Initialize 2D array bag[][]}
    \State{Initialize 2D array bag[][]}
    \EndProcedure{}
  \end{algorithmic}
  \caption{Dynamic Programming}
\end{algorithm}






\section{Experiments}
Analysis, evaluation, and critique of the algorithm and your
implementation. Include a description of the testing data you used and
a discussion of examples that illustrate major features of your
system. Testing is a critical part of system construction, and the
scope of your testing will be an important component in our
evaluation. Discuss what you learned from the implementation. \par

\vspace{2mm}

\noindent In order to compare the performance of each algorithm, we plotted a scatter plot of a percentage of totalValue / optimalValue for each algorithm as the number of items increase. We could see that the DP solution returns the optimal solution independent of the number of food items. For the greedy solution that uses the heuristic that just packs the items with the highest values does pretty poorly. This is because it is rarely the case that adding the highest valued items could still satisfy the macronutrient. Then in decreasing order, the algorithm would check if the item is under the specified limit. The result is a group of relatively few food items that \par
\vspace{2mm}
\noindent The greedy algorithm that takes into account the ratio of the item's value to the item's weight does pretty well.

\begin{table}
  \centering
  \begin{tabular}{ll}
    \toprule
    & Score \\
    \midrule
    Approach 1 & \\
    Approach 2 & \\
    \bottomrule
  \end{tabular}
  \caption{Description of the results.}
\end{table}


\subsection{Results}

 For algorithm-comparison projects: a section reporting empirical comparison results preferably presented graphically.\par
 \vspace{2mm}
 \noindent the DP solution ended up being the best


\section{Discussion}

Summary of approach and results. Major takeaways? Things you could improve in future work?\par 
\vspace{2mm}
\noindent eventually we want it to be able to take multiple constraints and multiple optimization functions


\appendix

\section{System Description}

 Appendix 1 – A clear description of how to use your system and how to generate the output you discussed in the write-up. \emph{The teaching staff must be able to run your system.}\par
 
 \noindent To test out the application, simple run 'python test.py'. to see how the different algorithms compare, run 'python compare.py'
 

\section{Group Makeup}

 Appendix 2 – A list of each project participant and that
participant’s contributions to the project. If the division of work
varies significantly from the project proposal, provide a brief
explanation.  Your code should be clearly documented. 



\bibliographystyle{plain} 
\bibliography{project-template}

\end{document}
